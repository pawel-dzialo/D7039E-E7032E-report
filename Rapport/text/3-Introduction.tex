\textit{A sense of unease, an increasing distance to the world around me, then a sudden lack of control, my body no longer obeys, soon the seizure will come.}


This is the feeling for some people who suffer from epileptic seizures, but for others, there is no warning, no foreboding feeling, only a sudden seizure. Epilepsy is a neurological disorder that affects around 50 million people worldwide \cite{WHODAT}.
A seizure can have many consequences for the afflicted individual. Fall injuries, risk of drowning, burns, or in rare cases, brain damage due to prolonged seizures. However, the fear of the event can be just as debilitating as the seizure itself. Many individuals with epilepsy suffer from anxiety or depression due to the knowledge of the afflictions. Many problems are highly preventable if the individual and their surroundings are prepared for the event. Given 2 minutes to prepare, the individual about to suffer an epileptic seizure can call a help contact, and lie or sit down, reducing the risk of injuries from falls to almost zero. This work aims to produce a method to provide that warning, ultimately improving the quality of life for those 50 million people with epilepsy. It is far from the first attempt. There has been much interest in epileptic prediction and warning devices in recent years. 

%kolla den rapport som sammanställer

%As far as the current devices go they mostly aim at detecting sudden and unsuspected movement, which 
The proposed system here is that of a wearable hat equipped with electrodes that can detect EEG signals, which then are fed through a neural network that detects a pre-ictal, a state that precedes the epileptic seizure. Hopefully, this hat can communicate with phone apps to send the afflicted individual's location to a designated emergency contact person. This work is therefore split into 3 main parts. The hardware/software interface of the hat and app, the neural network/prediction algorithm, and a preprocessing system that manages the data fed from the hat to the prediction algorithm.



%Inledningen introducerar läsaren till problemställningen och ger bakgrunden till problemet. Inledningen är viktig för det är här som läsaren skall ledas in i hur författaren tänkt. Hela avsnittet bör skrivas så att läsaren logiskt och motiverat leds fram till den problemställning som rapporten behandlar. I inledningen skall det finnas en översikt över närliggande tidigare arbeten inom ämnet. Även här gäller att göra läsaren intresserad så att hen läser vidare i rapporten. Slutklämmen i inledningen bör göras så att det blir en mjuk övergång från Inledning till nästa kapitel.



%Här kan vara exempel med numrerade referenser: ''kan beskrivas enligt \cite{Sterte2001}''.
%Ge syftet med arbetet dvs vad som vill åstadkommas med arbetet, frågeställningar och vilka avgränsningar som finns. Målen skall vara specifikt klarlagda samt i rapportens slutsatser ska det tydligt framgå hur målen uppnåtts.


%\subsection{Related work}


% Kan med fördel indelas i underrubriker; Bakgrund, Problemformulering, Litteraturstudie,
% Syfte och mål, Avgränsningar.
%
% Lite skriv-vett
%
% 1. Använd aktiv form (vattnet strömmade genom röret). Det gör rapporten mer livlig.
% 2. Använd dåtid för observationer mm. Exempelvis ”ökat tryck gav större flöde”.
% 3. Använd nutid för generaliseringar och allmänt giltiga påståenden. Exempelvis
% ”I de flesta fall tillhör problemen kategorin olösbara problem”.
% 4. Undvik strunt, pompösa meningar och alla överdrifter. Uttryck som ``utmärkt
% överenstämmelse'' eller ``fantastisk mätnoggrannhet'' får inte förekomma.
% 5. Samtliga tidigare arbeten som åberopas i rapporten skall refereras.
% 6. Skriv inte rapporten som en berättelse om vad ni gjort.
% 7. Berätta inte om de idéer som inte gav något.
% 8. Var mycket försiktig med negativa kommentarer om det egna arbetet.
