Inledningen introducerar läsaren till problemställningen och ger bakgrunden till problemet. Inledningen är viktig för det är här som läsaren skall ledas in i hur författaren tänkt. Hela avsnittet bör skrivas så att läsaren logiskt och motiverat leds fram till den problemställning som rapporten behandlar. I inledningen skall det finnas en översikt över närliggande tidigare arbeten inom ämnet. Även här gäller att göra läsaren intresserad så att hen läser vidare i rapporten. Slutklämmen i inledningen bör göras så att det blir en mjuk övergång från Inledning till nästa kapitel.



Här kan vara exempel med numrerade referenser: ''kan beskrivas enligt \cite{Sterte2001}''.
Ge syftet med arbetet dvs vad som vill åstadkommas med arbetet, frågeställningar och vilka avgränsningar som finns. Målen skall vara specifikt klarlagda samt i rapportens slutsatser ska det tydligt framgå hur målen uppnåtts.

% Kan med fördel indelas i underrubriker; Bakgrund, Problemformulering, Litteraturstudie,
% Syfte och mål, Avgränsningar.
%
% Lite skriv-vett
%
% 1. Använd aktiv form (vattnet strömmade genom röret). Det gör rapporten mer livlig.
% 2. Använd dåtid för observationer mm. Exempelvis ”ökat tryck gav större flöde”.
% 3. Använd nutid för generaliseringar och allmänt giltiga påståenden. Exempelvis
% ”I de flesta fall tillhör problemen kategorin olösbara problem”.
% 4. Undvik strunt, pompösa meningar och alla överdrifter. Uttryck som ``utmärkt
% överenstämmelse'' eller ``fantastisk mätnoggrannhet'' får inte förekomma.
% 5. Samtliga tidigare arbeten som åberopas i rapporten skall refereras.
% 6. Skriv inte rapporten som en berättelse om vad ni gjort.
% 7. Berätta inte om de idéer som inte gav något.
% 8. Var mycket försiktig med negativa kommentarer om det egna arbetet.
