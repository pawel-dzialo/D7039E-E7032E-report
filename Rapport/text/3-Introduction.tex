A sense of unease, a distance to world around me, then a sudden lack of control, my body no longer obeys, soon the seizure will come. This is the feeling for some people who suffer from epileptic seizure, but for some there is no warning, no foreboding feeling, only a sudden seizure. Epilepsy is a neurological disorder that is affecting around 50 million people world wide 
%[https://www.who.int/news-room/fact-sheets/detail/epilepsy#:~:text=Epilepsy]
A seizure can have many consequences for the individual afflicted, fall injuries,drowning, burns or in rare cases brain damage due to prolonged seizures. But the fear of the event can be just as depilitating as the seizure itself, many whit epilepsy suffers from anxiety or depression due to the knowledge of the afflictions. Many of these problems are highly preventable if the individual and its surrounding is prepared for the event. Given 2 minutes to prepare the afflicted(need better word) can call a loved one and lie or sit down, thus reducing the risk of falls to almost zero.In this work we aim to provide that warning to those 50 million, thus improving their quality of life. We are not the first to atempt this, there has been a great deal of interest in the area of epileptic prediction and warning in recent years. 

%kolla den rapport som sammanställer

%As far as the current devices go they mostly aim at detecting sudden and unsuspected movement, whic 
The system we propose is that of a wearable hat, equipped whit electrodes that can detect EEG signals, which then is runed through a Neural Network that detects the so called preictal state that precedes the epileptic seizure. hopefully this hat can communicate whit an app on your phone that can then send your location to a designated contact person. Thus this work is split into 3 main parts the Hardware/software interface of the hat and app, the neural network/prediction algoritm and a preprocessing system that manages the data fed from the hat to the prediction algorihtm.





%Inledningen introducerar läsaren till problemställningen och ger bakgrunden till problemet. Inledningen är viktig för det är här som läsaren skall ledas in i hur författaren tänkt. Hela avsnittet bör skrivas så att läsaren logiskt och motiverat leds fram till den problemställning som rapporten behandlar. I inledningen skall det finnas en översikt över närliggande tidigare arbeten inom ämnet. Även här gäller att göra läsaren intresserad så att hen läser vidare i rapporten. Slutklämmen i inledningen bör göras så att det blir en mjuk övergång från Inledning till nästa kapitel.



%Här kan vara exempel med numrerade referenser: ''kan beskrivas enligt \cite{Sterte2001}''.
%Ge syftet med arbetet dvs vad som vill åstadkommas med arbetet, frågeställningar och vilka avgränsningar som finns. Målen skall vara specifikt klarlagda samt i rapportens slutsatser ska det tydligt framgå hur målen uppnåtts.


\subsection{Related work}


% Kan med fördel indelas i underrubriker; Bakgrund, Problemformulering, Litteraturstudie,
% Syfte och mål, Avgränsningar.
%
% Lite skriv-vett
%
% 1. Använd aktiv form (vattnet strömmade genom röret). Det gör rapporten mer livlig.
% 2. Använd dåtid för observationer mm. Exempelvis ”ökat tryck gav större flöde”.
% 3. Använd nutid för generaliseringar och allmänt giltiga påståenden. Exempelvis
% ”I de flesta fall tillhör problemen kategorin olösbara problem”.
% 4. Undvik strunt, pompösa meningar och alla överdrifter. Uttryck som ``utmärkt
% överenstämmelse'' eller ``fantastisk mätnoggrannhet'' får inte förekomma.
% 5. Samtliga tidigare arbeten som åberopas i rapporten skall refereras.
% 6. Skriv inte rapporten som en berättelse om vad ni gjort.
% 7. Berätta inte om de idéer som inte gav något.
% 8. Var mycket försiktig med negativa kommentarer om det egna arbetet.
