Kan i vissa fall delas upp i metodbeskrivning, experimentell uppställning och arbetsgång. Att redogöra för sin metod är viktigt bland annat för att förklara varför den valda metoden ger ett tillförlitligt resultat. Alla antaganden och förenklingar måste anges och motiveras. Definiera matematiska modeller så att andra ingenjörer och forskare kan förstå vad du gjort.
Exempelvis utnyttjades Microsoft Excel 2013 för att analysera mätresultaten och plotta mätdata.

% Här beskrivs metoden, ofta är det lämpligt att dela upp texten i ett antal underrubriker.
% Använd alltid högst tre rubrik-nivåer.

\subsection{Experimentell uppställning}

Alla eventuella försöksuppställningar beskrivs på ett sådant sätt att andra kan upprepa samma försök och verifiera dina resultat. Utnyttja figurer som förenklar din beskrivning.
