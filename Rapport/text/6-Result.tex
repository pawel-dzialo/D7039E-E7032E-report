% Innehåll: resultat och analys.
% I vissa fall kan man ha ”Resultat och diskussion” som kapitel.

%Detta är förmodligen den största delen av rapporten. Här redovisas resultaten rakt på sak på ett objektivt/neutralt sätt. Ofta är det lämpligt att dela upp texten i ett antal underrubriker. Materialet måste presenteras i logisk ordning, vilket inte behöver vara den ordning i vilken försöket/arbetet har utförts.

%Läsaren skall kunna läsa rapporten utan att behöva bläddra fram och tillbaka. Det ska vara tydligt vad som är data respektive analys av data.
%Visas resultat i tabell- eller figurform så måste kortfattat beskrivas vad man ser i figurerna/tabellerna. De placeras i närheten (efter) där de först refererades.

%Som exempel visas fyra mätningar där variabel, 1, varierades. Resultat visas i tabell \ref{tvariabel123} nedan.

% Det skall alltid finnas en tabelltext som förklarar vad som finns i tabellen. Tabellnummer och text ska stå ovanför tabellen.

\begin{table}[ht]
\centering
    \begin{tabular}{c | c | c}
        \hline
        variabel 1 (s) & variabel 2 (m) & variabel 3 (J) \\
        \hline
        0,351 &	0,693 &	117 \\
        0,457 &	1,42 &	170 \\
        0,873 &	2,54 &	300 \\
        1,10 &	3,21 &	390 \\
        \hline
     \end{tabular} 
\caption{Förklarande text}
\label{tvariabel123}
\end{table}

%Med hjälp av mätvärdena i tabell 1 skapas en produktansats av typen potensfunktion
\begin{equation}
t = C L^\alpha \theta^\beta m^\gamma g^\delta
\end{equation}
% där $C, \alpha, ..., \delta$ är konstanter som ska bestämmas experimentellt.
% Avgiven värmemängd från brödrosten (variabel 3) som funktion av tiden variabel 1 visas i figur \ref{fvariabel3vs1}. Den linjära anpassningen i figuren visar att
\begin{equation}
{\rm (variabel \, 3)} = 351,8 \cdot {\rm (variabel \, 1)} - 0,3
\end{equation}
% och tillförd värmeeffekt till brödrosten bestäms då till 351,8 W.

% Diagram ska ha storhet och enhet på axlarna (SI). Är det tex logaritmerade diagram
% ska de ha storhet (men inte enhet) på axlarna. Figurnummer och text ska stå under
% figur och hänga ihop på samma sida.

\begin{figure}[ht]
\begin{center}
  \includegraphics[width=0.6\textwidth]{images/fig1.png}
  \caption{Exempelfigur som visar variabel 3 som funktion av variabel 1. \label{fvariabel3vs1}}
\end{center}
\end{figure}

\subsection{Underrubrik vid behov}

\subsubsection{Fler underrubriker om så behövs}
